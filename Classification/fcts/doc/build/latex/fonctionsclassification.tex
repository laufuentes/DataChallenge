%% Generated by Sphinx.
\def\sphinxdocclass{report}
\documentclass[letterpaper,10pt,french]{sphinxmanual}
\ifdefined\pdfpxdimen
   \let\sphinxpxdimen\pdfpxdimen\else\newdimen\sphinxpxdimen
\fi \sphinxpxdimen=.75bp\relax
\ifdefined\pdfimageresolution
    \pdfimageresolution= \numexpr \dimexpr1in\relax/\sphinxpxdimen\relax
\fi
%% let collapsible pdf bookmarks panel have high depth per default
\PassOptionsToPackage{bookmarksdepth=5}{hyperref}

\PassOptionsToPackage{warn}{textcomp}
\usepackage[utf8]{inputenc}
\ifdefined\DeclareUnicodeCharacter
% support both utf8 and utf8x syntaxes
  \ifdefined\DeclareUnicodeCharacterAsOptional
    \def\sphinxDUC#1{\DeclareUnicodeCharacter{"#1}}
  \else
    \let\sphinxDUC\DeclareUnicodeCharacter
  \fi
  \sphinxDUC{00A0}{\nobreakspace}
  \sphinxDUC{2500}{\sphinxunichar{2500}}
  \sphinxDUC{2502}{\sphinxunichar{2502}}
  \sphinxDUC{2514}{\sphinxunichar{2514}}
  \sphinxDUC{251C}{\sphinxunichar{251C}}
  \sphinxDUC{2572}{\textbackslash}
\fi
\usepackage{cmap}
\usepackage[T1]{fontenc}
\usepackage{amsmath,amssymb,amstext}
\usepackage{babel}



\usepackage{tgtermes}
\usepackage{tgheros}
\renewcommand{\ttdefault}{txtt}



\usepackage[Sonny]{fncychap}
\ChNameVar{\Large\normalfont\sffamily}
\ChTitleVar{\Large\normalfont\sffamily}
\usepackage{sphinx}

\fvset{fontsize=auto}
\usepackage{geometry}


% Include hyperref last.
\usepackage{hyperref}
% Fix anchor placement for figures with captions.
\usepackage{hypcap}% it must be loaded after hyperref.
% Set up styles of URL: it should be placed after hyperref.
\urlstyle{same}

\addto\captionsfrench{\renewcommand{\contentsname}{Contents:}}

\usepackage{sphinxmessages}
\setcounter{tocdepth}{1}



\title{Fonctions Classification}
\date{nov. 09, 2023}
\release{2023}
\author{Benali Nafissa et Fuentes Vicente Laura}
\newcommand{\sphinxlogo}{\vbox{}}
\renewcommand{\releasename}{Version}
\makeindex
\begin{document}

\ifdefined\shorthandoff
  \ifnum\catcode`\=\string=\active\shorthandoff{=}\fi
  \ifnum\catcode`\"=\active\shorthandoff{"}\fi
\fi

\pagestyle{empty}
\sphinxmaketitle
\pagestyle{plain}
\sphinxtableofcontents
\pagestyle{normal}
\phantomsection\label{\detokenize{index::doc}}


\sphinxstepscope


\chapter{general module}
\label{\detokenize{general:module-general}}\label{\detokenize{general:general-module}}\label{\detokenize{general::doc}}\index{module@\spxentry{module}!general@\spxentry{general}}\index{general@\spxentry{general}!module@\spxentry{module}}\index{CV\_rep() (dans le module general)@\spxentry{CV\_rep()}\spxextra{dans le module general}}

\begin{fulllineitems}
\phantomsection\label{\detokenize{general:general.CV_rep}}
\pysigstartsignatures
\pysiglinewithargsret{\sphinxcode{\sphinxupquote{general.}}\sphinxbfcode{\sphinxupquote{CV\_rep}}}{\emph{\DUrole{n}{Xtr}}, \emph{\DUrole{n}{ytr}}, \emph{\DUrole{n}{nfolds}}}{}
\pysigstopsignatures
\sphinxAtStartPar
La fonction crée des nouveaux jeux de données en sous\sphinxhyphen{}divisant les jeux de données en
\begin{quote}\begin{description}
\sphinxlineitem{Paramètres}\begin{itemize}
\item {} 
\sphinxAtStartPar
\sphinxstyleliteralstrong{\sphinxupquote{Xtr}} (\sphinxstyleliteralemphasis{\sphinxupquote{pd.DataFrame}}) \textendash{} Jeu de données à diviser, contenant les co\sphinxhyphen{}variables

\item {} 
\sphinxAtStartPar
\sphinxstyleliteralstrong{\sphinxupquote{ytr}} (\sphinxstyleliteralemphasis{\sphinxupquote{pd.DataFrame}}) \textendash{} Vecteur à diviser, contenant la variable à prédire

\item {} 
\sphinxAtStartPar
\sphinxstyleliteralstrong{\sphinxupquote{nfolds}} (\sphinxstyleliteralemphasis{\sphinxupquote{int}}) \textendash{} Nombre representant en combien de sous dataframes on souhaite diviser Xtr et ytr

\end{itemize}

\sphinxlineitem{Renvoie}
\sphinxAtStartPar
liste contenant (n\sphinxhyphen{}folds) jeux de données
y\_new (list): liste contenant (les n\sphinxhyphen{}folds) nouvelles version de ytr

\sphinxlineitem{Type renvoyé}
\sphinxAtStartPar
X\_new (list)

\end{description}\end{quote}

\end{fulllineitems}

\index{Label\_Encode() (dans le module general)@\spxentry{Label\_Encode()}\spxextra{dans le module general}}

\begin{fulllineitems}
\phantomsection\label{\detokenize{general:general.Label_Encode}}
\pysigstartsignatures
\pysiglinewithargsret{\sphinxcode{\sphinxupquote{general.}}\sphinxbfcode{\sphinxupquote{Label\_Encode}}}{\emph{\DUrole{n}{y\_tr}}, \emph{\DUrole{n}{new\_pred}}}{}
\pysigstopsignatures
\sphinxAtStartPar
Fonction qui normalise les etiquettes de plusieurs jeux de données
\begin{quote}\begin{description}
\sphinxlineitem{Paramètres}\begin{itemize}
\item {} 
\sphinxAtStartPar
\sphinxstyleliteralstrong{\sphinxupquote{y\_tr}} (\sphinxstyleliteralemphasis{\sphinxupquote{pd.DataFrame}}) \textendash{} vecteur qui contient les etiquettes que l’on utilisera pour entrainer le modèle

\item {} 
\sphinxAtStartPar
\sphinxstyleliteralstrong{\sphinxupquote{new\_pred}} (\sphinxstyleliteralemphasis{\sphinxupquote{np.array}}) \textendash{} vecteur auquel on appliquera la normalisation des etiquettes

\end{itemize}

\sphinxlineitem{Renvoie}
\sphinxAtStartPar
même vecteur qu’avant avec les etiquettes normalisées

\sphinxlineitem{Type renvoyé}
\sphinxAtStartPar
new\_pred (np.array)

\end{description}\end{quote}

\end{fulllineitems}

\index{submission() (dans le module general)@\spxentry{submission()}\spxextra{dans le module general}}

\begin{fulllineitems}
\phantomsection\label{\detokenize{general:general.submission}}
\pysigstartsignatures
\pysiglinewithargsret{\sphinxcode{\sphinxupquote{general.}}\sphinxbfcode{\sphinxupquote{submission}}}{\emph{\DUrole{n}{new\_pred}}, \emph{\DUrole{n}{name}}, \emph{\DUrole{n}{date}}, \emph{\DUrole{n}{X\_test}}}{}
\pysigstopsignatures
\sphinxAtStartPar
Fonction qui genere un csv avec les valeurs à prédire associées à ses respectifs identificateurs
Attention: il faudrait aussi verifier que obj\_ID se retrouve dans le repertoire nommé
\begin{quote}\begin{description}
\sphinxlineitem{Paramètres}\begin{itemize}
\item {} 
\sphinxAtStartPar
\sphinxstyleliteralstrong{\sphinxupquote{new\_pred}} (\sphinxstyleliteralemphasis{\sphinxupquote{np.array}}) \textendash{} vecteur contenant les predictions

\item {} 
\sphinxAtStartPar
\sphinxstyleliteralstrong{\sphinxupquote{name}} (\sphinxstyleliteralemphasis{\sphinxupquote{string}}) \textendash{} méthode utilisée pour generer la prédiction (ex: “LDA”)

\item {} 
\sphinxAtStartPar
\sphinxstyleliteralstrong{\sphinxupquote{date}} (\sphinxstyleliteralemphasis{\sphinxupquote{string}}) \textendash{} date de la prédiction (ex: “04/10”)

\item {} 
\sphinxAtStartPar
\sphinxstyleliteralstrong{\sphinxupquote{X\_test}} (\sphinxstyleliteralemphasis{\sphinxupquote{pd.DataFrame}}) \textendash{} Jeu de données test utilisé pour calculer la prédiction

\end{itemize}

\end{description}\end{quote}

\end{fulllineitems}


\sphinxstepscope


\chapter{KNN\_sel\_var module}
\label{\detokenize{KNN_sel_var:module-KNN_sel_var}}\label{\detokenize{KNN_sel_var:knn-sel-var-module}}\label{\detokenize{KNN_sel_var::doc}}\index{module@\spxentry{module}!KNN\_sel\_var@\spxentry{KNN\_sel\_var}}\index{KNN\_sel\_var@\spxentry{KNN\_sel\_var}!module@\spxentry{module}}\index{choix\_var\_knn() (dans le module KNN\_sel\_var)@\spxentry{choix\_var\_knn()}\spxextra{dans le module KNN\_sel\_var}}

\begin{fulllineitems}
\phantomsection\label{\detokenize{KNN_sel_var:KNN_sel_var.choix_var_knn}}
\pysigstartsignatures
\pysiglinewithargsret{\sphinxcode{\sphinxupquote{KNN\_sel\_var.}}\sphinxbfcode{\sphinxupquote{choix\_var\_knn}}}{\emph{\DUrole{n}{X\_tr}}, \emph{\DUrole{n}{y\_tr}}, \emph{\DUrole{n}{vars}}, \emph{\DUrole{n}{columns}}}{}
\pysigstopsignatures
\sphinxAtStartPar
Fonction qui calcule pour un vecteur de variables, le f1\_score (par cross\sphinxhyphen{}validation) associé à l’ajout d’une nouvelle variable.
Cette dernière choisit une nouvelle variable parmi cols et la rajoute sur vars.
\begin{quote}\begin{description}
\sphinxlineitem{Paramètres}\begin{itemize}
\item {} 
\sphinxAtStartPar
\sphinxstyleliteralstrong{\sphinxupquote{X\_tr}} (\sphinxstyleliteralemphasis{\sphinxupquote{pd.DataFrame}}) \textendash{} Jeu de données d’entraînement avec les co\sphinxhyphen{}variables

\item {} 
\sphinxAtStartPar
\sphinxstyleliteralstrong{\sphinxupquote{y\_tr}} (\sphinxstyleliteralemphasis{\sphinxupquote{pd.DataFrame ou np.array}}) \textendash{} Jeu de données d’entraînement avec la variable à prédire

\item {} 
\sphinxAtStartPar
\sphinxstyleliteralstrong{\sphinxupquote{vars}} (\sphinxstyleliteralemphasis{\sphinxupquote{pd.Index}}) \textendash{} vecteur avec les variables déjà selectionnées auparavant

\item {} 
\sphinxAtStartPar
\sphinxstyleliteralstrong{\sphinxupquote{columns}} (\sphinxstyleliteralemphasis{\sphinxupquote{pd.Index}}) \textendash{} vecteur avec des nouvelles variables non inclues dans vars

\end{itemize}

\sphinxlineitem{Renvoie}
\sphinxAtStartPar
retourne le vecteurs vars avec la nouvelle variable maxiimisant le f1\_score
cols (pd.Index): retourne le vecteur de nouvelles co\sphinxhyphen{}variables sans la variable finalement choisie

\sphinxlineitem{Type renvoyé}
\sphinxAtStartPar
vars (pd.Index)

\end{description}\end{quote}

\end{fulllineitems}

\index{main() (dans le module KNN\_sel\_var)@\spxentry{main()}\spxextra{dans le module KNN\_sel\_var}}

\begin{fulllineitems}
\phantomsection\label{\detokenize{KNN_sel_var:KNN_sel_var.main}}
\pysigstartsignatures
\pysiglinewithargsret{\sphinxcode{\sphinxupquote{KNN\_sel\_var.}}\sphinxbfcode{\sphinxupquote{main}}}{\emph{\DUrole{n}{X\_tr}}, \emph{\DUrole{n}{y\_tr}}, \emph{\DUrole{n}{X\_te}}, \emph{\DUrole{n}{y\_te}}}{}
\pysigstopsignatures
\sphinxAtStartPar
Fonction qui effectue la selection de variables
\begin{quote}\begin{description}
\sphinxlineitem{Paramètres}\begin{itemize}
\item {} 
\sphinxAtStartPar
\sphinxstyleliteralstrong{\sphinxupquote{X\_tr}} (\sphinxstyleliteralemphasis{\sphinxupquote{pd.DataFrame}}) \textendash{} Jeu de données d’entraînement avec les co\sphinxhyphen{}variables

\item {} 
\sphinxAtStartPar
\sphinxstyleliteralstrong{\sphinxupquote{y\_tr}} (\sphinxstyleliteralemphasis{\sphinxupquote{pd.DataFrame ou np.array}}) \textendash{} Jeu de données d’entraînement avec la variable à prédire

\item {} 
\sphinxAtStartPar
\sphinxstyleliteralstrong{\sphinxupquote{X\_te}} (\sphinxstyleliteralemphasis{\sphinxupquote{pd.DataFrame}}) \textendash{} Jeu de données test avec les co\sphinxhyphen{}variables

\item {} 
\sphinxAtStartPar
\sphinxstyleliteralstrong{\sphinxupquote{y\_te}} (\sphinxstyleliteralemphasis{\sphinxupquote{pd.DataFrame ou np.array}}) \textendash{} Jeu de données tets avec la variable à prédire

\end{itemize}

\sphinxlineitem{Renvoie}
\sphinxAtStartPar
vecteur contenant le choix des variables aménant au meilleur f1\_score par cross validation

\sphinxlineitem{Type renvoyé}
\sphinxAtStartPar
choix\_vars (pd.Index)

\end{description}\end{quote}

\end{fulllineitems}

\index{plot\_vars() (dans le module KNN\_sel\_var)@\spxentry{plot\_vars()}\spxextra{dans le module KNN\_sel\_var}}

\begin{fulllineitems}
\phantomsection\label{\detokenize{KNN_sel_var:KNN_sel_var.plot_vars}}
\pysigstartsignatures
\pysiglinewithargsret{\sphinxcode{\sphinxupquote{KNN\_sel\_var.}}\sphinxbfcode{\sphinxupquote{plot\_vars}}}{\emph{\DUrole{n}{res\_plot}}, \emph{\DUrole{n}{vars}}}{}
\pysigstopsignatures
\sphinxAtStartPar
Fonction qui renvoit un plot avec les f1\_scores associés aux prédicteurs issus de l’ajout itératif de chaque variable
\begin{quote}\begin{description}
\sphinxlineitem{Paramètres}\begin{itemize}
\item {} 
\sphinxAtStartPar
\sphinxstyleliteralstrong{\sphinxupquote{res\_plot}} (\sphinxstyleliteralemphasis{\sphinxupquote{list}}) \textendash{} liste contenant les scores f1 associés à l’ajout de chaque variable

\item {} 
\sphinxAtStartPar
\sphinxstyleliteralstrong{\sphinxupquote{vars}} (\sphinxstyleliteralemphasis{\sphinxupquote{pd.Index}}) \textendash{} vecteur contenant les variables ajoutées dans l’ordre

\end{itemize}

\end{description}\end{quote}

\end{fulllineitems}

\index{pred\_acc\_var() (dans le module KNN\_sel\_var)@\spxentry{pred\_acc\_var()}\spxextra{dans le module KNN\_sel\_var}}

\begin{fulllineitems}
\phantomsection\label{\detokenize{KNN_sel_var:KNN_sel_var.pred_acc_var}}
\pysigstartsignatures
\pysiglinewithargsret{\sphinxcode{\sphinxupquote{KNN\_sel\_var.}}\sphinxbfcode{\sphinxupquote{pred\_acc\_var}}}{\emph{\DUrole{n}{vars}}, \emph{\DUrole{n}{X\_tr}}, \emph{\DUrole{n}{y\_tr}}, \emph{\DUrole{n}{X\_te}}, \emph{\DUrole{n}{y\_te}}}{}
\pysigstopsignatures
\sphinxAtStartPar
Fonction qui pour des variables données, enmtraîne une knn et calcule l’erreur test associé.
\begin{quote}\begin{description}
\sphinxlineitem{Paramètres}\begin{itemize}
\item {} 
\sphinxAtStartPar
\sphinxstyleliteralstrong{\sphinxupquote{vars}} (\sphinxstyleliteralemphasis{\sphinxupquote{pd.Index}}) \textendash{} variables selectionnées

\item {} 
\sphinxAtStartPar
\sphinxstyleliteralstrong{\sphinxupquote{X\_tr}} (\sphinxstyleliteralemphasis{\sphinxupquote{pd.DataFrame}}) \textendash{} Jeu de données d’entraînement avec les co\sphinxhyphen{}variables

\item {} 
\sphinxAtStartPar
\sphinxstyleliteralstrong{\sphinxupquote{y\_tr}} (\sphinxstyleliteralemphasis{\sphinxupquote{pd.DataFrame ou np.array}}) \textendash{} Jeu de données d’entraînement avec la variable à prédire

\item {} 
\sphinxAtStartPar
\sphinxstyleliteralstrong{\sphinxupquote{X\_te}} (\sphinxstyleliteralemphasis{\sphinxupquote{pd.DataFrame}}) \textendash{} Jeu de données test avec les co\sphinxhyphen{}variables

\item {} 
\sphinxAtStartPar
\sphinxstyleliteralstrong{\sphinxupquote{y\_te}} (\sphinxstyleliteralemphasis{\sphinxupquote{pd.DataFrame ou np.array}}) \textendash{} Jeu de données test avec la variable à prédire

\end{itemize}

\sphinxlineitem{Renvoie}
\sphinxAtStartPar
score f1 sur le jeu de données test

\sphinxlineitem{Type renvoyé}
\sphinxAtStartPar
f1\_score(y\_te, pred, average= »weighted »)

\end{description}\end{quote}

\end{fulllineitems}

\index{train() (dans le module KNN\_sel\_var)@\spxentry{train()}\spxextra{dans le module KNN\_sel\_var}}

\begin{fulllineitems}
\phantomsection\label{\detokenize{KNN_sel_var:KNN_sel_var.train}}
\pysigstartsignatures
\pysiglinewithargsret{\sphinxcode{\sphinxupquote{KNN\_sel\_var.}}\sphinxbfcode{\sphinxupquote{train}}}{\emph{\DUrole{n}{X}}, \emph{\DUrole{n}{y}}, \emph{\DUrole{n}{X\_test}}, \emph{\DUrole{n}{vars}}}{}
\pysigstopsignatures
\sphinxAtStartPar
Fonction qui entraîne le modèle pour les variables choisies et crée une prédiction
\begin{quote}\begin{description}
\sphinxlineitem{Paramètres}\begin{itemize}
\item {} 
\sphinxAtStartPar
\sphinxstyleliteralstrong{\sphinxupquote{X}} (\sphinxstyleliteralemphasis{\sphinxupquote{pd.DataFrame}}) \textendash{} Jeu de données d’entraînement avec les co\sphinxhyphen{}variables

\item {} 
\sphinxAtStartPar
\sphinxstyleliteralstrong{\sphinxupquote{y}} (\sphinxstyleliteralemphasis{\sphinxupquote{pd.DataFrame ou np.array}}) \textendash{} Jeu de données d’entraînement avec la variable à prédire

\item {} 
\sphinxAtStartPar
\sphinxstyleliteralstrong{\sphinxupquote{X\_test}} (\sphinxstyleliteralemphasis{\sphinxupquote{pdDataFrame}}) \textendash{} Jeu de données test avec les co\sphinxhyphen{}variables

\item {} 
\sphinxAtStartPar
\sphinxstyleliteralstrong{\sphinxupquote{vars}} (\sphinxstyleliteralemphasis{\sphinxupquote{pd.Index}}) \textendash{} vecteur contenant les variables choisies au préalable

\end{itemize}

\sphinxlineitem{Renvoie}
\sphinxAtStartPar
\_description\_

\sphinxlineitem{Type renvoyé}
\sphinxAtStartPar
\_type\_

\end{description}\end{quote}

\end{fulllineitems}


\sphinxstepscope


\chapter{methode\_fait\_maison module}
\label{\detokenize{methode_fait_maison:module-methode_fait_maison}}\label{\detokenize{methode_fait_maison:methode-fait-maison-module}}\label{\detokenize{methode_fait_maison::doc}}\index{module@\spxentry{module}!methode\_fait\_maison@\spxentry{methode\_fait\_maison}}\index{methode\_fait\_maison@\spxentry{methode\_fait\_maison}!module@\spxentry{module}}\index{choix\_seuils() (dans le module methode\_fait\_maison)@\spxentry{choix\_seuils()}\spxextra{dans le module methode\_fait\_maison}}

\begin{fulllineitems}
\phantomsection\label{\detokenize{methode_fait_maison:methode_fait_maison.choix_seuils}}
\pysigstartsignatures
\pysiglinewithargsret{\sphinxcode{\sphinxupquote{methode\_fait\_maison.}}\sphinxbfcode{\sphinxupquote{choix\_seuils}}}{\emph{\DUrole{n}{X}}, \emph{\DUrole{n}{y}}, \emph{\DUrole{n}{folds}}, \emph{\DUrole{n}{nb\_seuils}}, \emph{\DUrole{n}{n\_var}}}{}
\pysigstopsignatures
\sphinxAtStartPar
Fonction qui choisi les seuils en fonction des résultats de cross validation
\begin{quote}\begin{description}
\sphinxlineitem{Paramètres}\begin{itemize}
\item {} 
\sphinxAtStartPar
\sphinxstyleliteralstrong{\sphinxupquote{X}} (\sphinxstyleliteralemphasis{\sphinxupquote{pd.DataFrame}}) \textendash{} Jeu de données avec les co\sphinxhyphen{}variables sur lesquel on veut entrainer les valeurs seuils

\item {} 
\sphinxAtStartPar
\sphinxstyleliteralstrong{\sphinxupquote{y}} (\sphinxstyleliteralemphasis{\sphinxupquote{pd.DataFrame}}) \textendash{} Jeu de données avec la variable à prédire sur lequel on veut entrainer les valeurs seuils

\item {} 
\sphinxAtStartPar
\sphinxstyleliteralstrong{\sphinxupquote{folds}} (\sphinxstyleliteralemphasis{\sphinxupquote{int}}) \textendash{} nombre de folds (sous\sphinxhyphen{}divisions) pour faire la cross\sphinxhyphen{}validation

\item {} 
\sphinxAtStartPar
\sphinxstyleliteralstrong{\sphinxupquote{nb\_seuils}} (\sphinxstyleliteralemphasis{\sphinxupquote{int}}) \textendash{} nombre de seuils à choisir

\item {} 
\sphinxAtStartPar
\sphinxstyleliteralstrong{\sphinxupquote{n\_var}} (\sphinxstyleliteralemphasis{\sphinxupquote{string}}) \textendash{} nom de la variable sur laquelle on travaille

\end{itemize}

\sphinxlineitem{Renvoie}
\sphinxAtStartPar
seuil choisi pour distinguer les classes 1 et 0
seuil\_2 (float): seuil choisi pour distinguer les classes 0 et 2

\sphinxlineitem{Type renvoyé}
\sphinxAtStartPar
seuil\_0 (float)

\end{description}\end{quote}

\end{fulllineitems}

\index{f1() (dans le module methode\_fait\_maison)@\spxentry{f1()}\spxextra{dans le module methode\_fait\_maison}}

\begin{fulllineitems}
\phantomsection\label{\detokenize{methode_fait_maison:methode_fait_maison.f1}}
\pysigstartsignatures
\pysiglinewithargsret{\sphinxcode{\sphinxupquote{methode\_fait\_maison.}}\sphinxbfcode{\sphinxupquote{f1}}}{\emph{\DUrole{n}{ytr}}, \emph{\DUrole{n}{pred}}, \emph{\DUrole{n}{lab}}}{}
\pysigstopsignatures
\sphinxAtStartPar
Fonction qui calcule le score f1 associé à une prédiction (pour un problème de classification binaire, C={[}3,lab{]} avec lab=\{0,2\})
\begin{quote}\begin{description}
\sphinxlineitem{Paramètres}\begin{itemize}
\item {} 
\sphinxAtStartPar
\sphinxstyleliteralstrong{\sphinxupquote{ytr}} (\sphinxstyleliteralemphasis{\sphinxupquote{pd.DataFrame}}) \textendash{} vecteur contenant les labels du jeu de données train

\item {} 
\sphinxAtStartPar
\sphinxstyleliteralstrong{\sphinxupquote{pred}} (\sphinxstyleliteralemphasis{\sphinxupquote{np.array}}) \textendash{} vecteur contenant des prédictions (2 classes)

\item {} 
\sphinxAtStartPar
\sphinxstyleliteralstrong{\sphinxupquote{lab}} (\sphinxstyleliteralemphasis{\sphinxupquote{\_type\_}}) \textendash{} \_description\_

\end{itemize}

\sphinxlineitem{Renvoie}
\sphinxAtStartPar
valeur du score f1 associé à cette prédiction

\sphinxlineitem{Type renvoyé}
\sphinxAtStartPar
(2*TP)/(2*TP + FP + FN) (float)

\end{description}\end{quote}

\end{fulllineitems}

\index{frontiere() (dans le module methode\_fait\_maison)@\spxentry{frontiere()}\spxextra{dans le module methode\_fait\_maison}}

\begin{fulllineitems}
\phantomsection\label{\detokenize{methode_fait_maison:methode_fait_maison.frontiere}}
\pysigstartsignatures
\pysiglinewithargsret{\sphinxcode{\sphinxupquote{methode\_fait\_maison.}}\sphinxbfcode{\sphinxupquote{frontiere}}}{\emph{\DUrole{n}{Xtr}}, \emph{\DUrole{n}{ytr}}, \emph{\DUrole{n}{lab}}, \emph{\DUrole{n}{folds}}, \emph{\DUrole{n}{nb\_seuils}}, \emph{\DUrole{n}{n\_var}}}{}
\pysigstopsignatures
\sphinxAtStartPar
Cette fonction crée un vecteur contenant, nb\_seuils, seuils différents et teste leur performance (f1 score)
pour classfifier le label lab à partir de la variable n\_var. On utilisera la méthode de cross validation pour tester les performances
moyennes de chaque seuil.
\begin{quote}\begin{description}
\sphinxlineitem{Paramètres}\begin{itemize}
\item {} 
\sphinxAtStartPar
\sphinxstyleliteralstrong{\sphinxupquote{Xtr}} (\sphinxstyleliteralemphasis{\sphinxupquote{pd.DataFrame}}) \textendash{} Jeu de données train contenant les co\sphinxhyphen{}variables

\item {} 
\sphinxAtStartPar
\sphinxstyleliteralstrong{\sphinxupquote{ytr}} (\sphinxstyleliteralemphasis{\sphinxupquote{pd.DataFrame}}) \textendash{} Jeu de données train contenant la variable à prédire

\item {} 
\sphinxAtStartPar
\sphinxstyleliteralstrong{\sphinxupquote{(}}\sphinxstyleliteralstrong{\sphinxupquote{int}} (\sphinxstyleliteralemphasis{\sphinxupquote{lab}}) \textendash{} 0 ou 2): label pour lequel calculer la valeur frontière

\item {} 
\sphinxAtStartPar
\sphinxstyleliteralstrong{\sphinxupquote{folds}} (\sphinxstyleliteralemphasis{\sphinxupquote{int}}) \textendash{} nombre de folds a créer pour performer la cross\sphinxhyphen{}validation

\item {} 
\sphinxAtStartPar
\sphinxstyleliteralstrong{\sphinxupquote{nb\_seuils}} (\sphinxstyleliteralemphasis{\sphinxupquote{int}}) \textendash{} nombre de seuils à tester entre la valeur min,max de n\_var

\item {} 
\sphinxAtStartPar
\sphinxstyleliteralstrong{\sphinxupquote{n\_var}} (\sphinxstyleliteralemphasis{\sphinxupquote{string}}) \textendash{} nom de la variable sur laquelle on veut calculer le seuil

\end{itemize}

\sphinxlineitem{Renvoie}
\sphinxAtStartPar
vecteur contenant les f1\sphinxhyphen{}scores moyennes de chaque seuil par cross\sphinxhyphen{}validation

\sphinxlineitem{Type renvoyé}
\sphinxAtStartPar
results.mean(axis=0) (np.array)

\end{description}\end{quote}

\end{fulllineitems}

\index{melange() (dans le module methode\_fait\_maison)@\spxentry{melange()}\spxextra{dans le module methode\_fait\_maison}}

\begin{fulllineitems}
\phantomsection\label{\detokenize{methode_fait_maison:methode_fait_maison.melange}}
\pysigstartsignatures
\pysiglinewithargsret{\sphinxcode{\sphinxupquote{methode\_fait\_maison.}}\sphinxbfcode{\sphinxupquote{melange}}}{\emph{\DUrole{n}{folds}}, \emph{\DUrole{n}{Xtr}}, \emph{\DUrole{n}{ytr}}, \emph{\DUrole{n}{var1}}, \emph{\DUrole{n}{seuil1}}, \emph{\DUrole{n}{model2}}, \emph{\DUrole{n}{vars}}}{}
\pysigstopsignatures
\sphinxAtStartPar
Fonction qui choisi la probabilité p (avec cross\sphinxhyphen{}validation) que l’on tirera sur une bernouilli pour mélanger deux prédictions
\begin{quote}\begin{description}
\sphinxlineitem{Paramètres}\begin{itemize}
\item {} 
\sphinxAtStartPar
\sphinxstyleliteralstrong{\sphinxupquote{folds}} (\sphinxstyleliteralemphasis{\sphinxupquote{int}}) \textendash{} nombre de

\item {} 
\sphinxAtStartPar
\sphinxstyleliteralstrong{\sphinxupquote{Xtr}} (\sphinxstyleliteralemphasis{\sphinxupquote{np.array ou pd.DataFrame}}) \textendash{} Vecteur d’entrainement contenant les co\sphinxhyphen{}variables

\item {} 
\sphinxAtStartPar
\sphinxstyleliteralstrong{\sphinxupquote{ytr}} (\sphinxstyleliteralemphasis{\sphinxupquote{np.array ou pd.DataFrame}}) \textendash{} Vecteur d’entrainement contenant la variable a predire

\item {} 
\sphinxAtStartPar
\sphinxstyleliteralstrong{\sphinxupquote{var1}} (\sphinxstyleliteralemphasis{\sphinxupquote{string}}) \textendash{} variable sur laquelle se basent les seuil1

\item {} 
\sphinxAtStartPar
\sphinxstyleliteralstrong{\sphinxupquote{seuil1}} (\sphinxstyleliteralemphasis{\sphinxupquote{list}}) \textendash{} liste avec les deux seuils choisis avec la méthode basée sur var1 (de la forme {[}seuil\_0,seuil\_2{]})

\item {} 
\sphinxAtStartPar
\sphinxstyleliteralstrong{\sphinxupquote{model2}} \textendash{} modèle 2 entraîné

\end{itemize}

\sphinxlineitem{Renvoie}
\sphinxAtStartPar
probabilité que l’on utilisera pour mélanger deux predictions

\sphinxlineitem{Type renvoyé}
\sphinxAtStartPar
proba

\end{description}\end{quote}

\end{fulllineitems}

\index{pred\_mel() (dans le module methode\_fait\_maison)@\spxentry{pred\_mel()}\spxextra{dans le module methode\_fait\_maison}}

\begin{fulllineitems}
\phantomsection\label{\detokenize{methode_fait_maison:methode_fait_maison.pred_mel}}
\pysigstartsignatures
\pysiglinewithargsret{\sphinxcode{\sphinxupquote{methode\_fait\_maison.}}\sphinxbfcode{\sphinxupquote{pred\_mel}}}{\emph{\DUrole{n}{pred1}}, \emph{\DUrole{n}{pred2}}, \emph{\DUrole{n}{p}}}{}
\pysigstopsignatures
\sphinxAtStartPar
Fonction qui mélange deux prédictions avec probabilité p sur une Bernouilli
\begin{quote}\begin{description}
\sphinxlineitem{Paramètres}\begin{itemize}
\item {} 
\sphinxAtStartPar
\sphinxstyleliteralstrong{\sphinxupquote{pred1}} (\sphinxstyleliteralemphasis{\sphinxupquote{np.array}}) \textendash{} predicteur 1 (à mélanger)

\item {} 
\sphinxAtStartPar
\sphinxstyleliteralstrong{\sphinxupquote{pred2}} (\sphinxstyleliteralemphasis{\sphinxupquote{np.array}}) \textendash{} prédicteur 2 (à mélanger)

\item {} 
\sphinxAtStartPar
\sphinxstyleliteralstrong{\sphinxupquote{p}} (\sphinxstyleliteralemphasis{\sphinxupquote{np.float}}) \textendash{} probabilité pour la Bernouilli

\end{itemize}

\sphinxlineitem{Renvoie}
\sphinxAtStartPar
nouvelle prédiction contenant le mélange des deux prédicteurs

\sphinxlineitem{Type renvoyé}
\sphinxAtStartPar
tirage(p,p1,p2) (np.array)

\end{description}\end{quote}

\end{fulllineitems}

\index{predict() (dans le module methode\_fait\_maison)@\spxentry{predict()}\spxextra{dans le module methode\_fait\_maison}}

\begin{fulllineitems}
\phantomsection\label{\detokenize{methode_fait_maison:methode_fait_maison.predict}}
\pysigstartsignatures
\pysiglinewithargsret{\sphinxcode{\sphinxupquote{methode\_fait\_maison.}}\sphinxbfcode{\sphinxupquote{predict}}}{\emph{\DUrole{n}{val0}}, \emph{\DUrole{n}{val2}}, \emph{\DUrole{n}{X\_te}}, \emph{\DUrole{n}{n\_var}}}{}
\pysigstopsignatures
\sphinxAtStartPar
Crée notre prédiction en fonction des valeurs val0 et val2 calculées précédamment.
Il crée un vecteur prédiction avec des valeurs:
\sphinxhyphen{} 1: n\_var appartenant à {]}\sphinxhyphen{}inf,val0{[}
\sphinxhyphen{} 0: n\_var appartenant à {[}val0,val2{[}
\sphinxhyphen{} 2: n\_var appartenant à {[}val2,+inf{[}
\begin{quote}\begin{description}
\sphinxlineitem{Paramètres}\begin{itemize}
\item {} 
\sphinxAtStartPar
\sphinxstyleliteralstrong{\sphinxupquote{val0}} (\sphinxstyleliteralemphasis{\sphinxupquote{float}}) \textendash{} valeur seuil calculée pour distinguer la classe 1 et 0

\item {} 
\sphinxAtStartPar
\sphinxstyleliteralstrong{\sphinxupquote{val2}} (\sphinxstyleliteralemphasis{\sphinxupquote{float}}) \textendash{} valeur seuil calculée pour distinguer la classe 0 et 2

\item {} 
\sphinxAtStartPar
\sphinxstyleliteralstrong{\sphinxupquote{X\_te}} (\sphinxstyleliteralemphasis{\sphinxupquote{pd.DataFrame}}) \textendash{} vecteur des covariables test sur lesquels on va regarder la valeur de n\_var

\item {} 
\sphinxAtStartPar
\sphinxstyleliteralstrong{\sphinxupquote{n\_var}} (\sphinxstyleliteralemphasis{\sphinxupquote{string}}) \textendash{} nom de la variable sur laquelle on base la prédiction

\end{itemize}

\sphinxlineitem{Renvoie}
\sphinxAtStartPar
vecteur contenant les prédictions de notre méthode\_seuil

\sphinxlineitem{Type renvoyé}
\sphinxAtStartPar
pred

\end{description}\end{quote}

\end{fulllineitems}

\index{tirage() (dans le module methode\_fait\_maison)@\spxentry{tirage()}\spxextra{dans le module methode\_fait\_maison}}

\begin{fulllineitems}
\phantomsection\label{\detokenize{methode_fait_maison:methode_fait_maison.tirage}}
\pysigstartsignatures
\pysiglinewithargsret{\sphinxcode{\sphinxupquote{methode\_fait\_maison.}}\sphinxbfcode{\sphinxupquote{tirage}}}{\emph{\DUrole{n}{p}}, \emph{\DUrole{n}{pred1}}, \emph{\DUrole{n}{pred2}}}{}
\pysigstopsignatures
\sphinxAtStartPar
Fonction qui tire melange deux vecteurs de probabilités. Elle repere les indices dans lequels les deux predicitions
sont differentes et choisi une des deux valeurs en tirant aleatoirement une bernouilli avec proba p.
\begin{quote}\begin{description}
\sphinxlineitem{Paramètres}\begin{itemize}
\item {} 
\sphinxAtStartPar
\sphinxstyleliteralstrong{\sphinxupquote{p}} (\sphinxstyleliteralemphasis{\sphinxupquote{np.float}}) \textendash{} probabilité associé au tirage aléatoire de la loi de Bernouilli

\item {} 
\sphinxAtStartPar
\sphinxstyleliteralstrong{\sphinxupquote{pred1}} (\sphinxstyleliteralemphasis{\sphinxupquote{np.array}}) \textendash{} Vecteur contenant les prédictions de la méthode 1

\item {} 
\sphinxAtStartPar
\sphinxstyleliteralstrong{\sphinxupquote{pred2}} (\sphinxstyleliteralemphasis{\sphinxupquote{np.array}}) \textendash{} Vecteur contenant les prédictions de la méthode 1

\end{itemize}

\sphinxlineitem{Renvoie}
\sphinxAtStartPar
nouveau vecteur de prédictions calculé à partir des deux autres prédictions

\sphinxlineitem{Type renvoyé}
\sphinxAtStartPar
res

\end{description}\end{quote}

\end{fulllineitems}



\chapter{Indices and tables}
\label{\detokenize{index:indices-and-tables}}\begin{itemize}
\item {} 
\sphinxAtStartPar
\DUrole{xref,std,std-ref}{genindex}

\item {} 
\sphinxAtStartPar
\DUrole{xref,std,std-ref}{modindex}

\item {} 
\sphinxAtStartPar
\DUrole{xref,std,std-ref}{search}

\end{itemize}


\renewcommand{\indexname}{Index des modules Python}
\begin{sphinxtheindex}
\let\bigletter\sphinxstyleindexlettergroup
\bigletter{g}
\item\relax\sphinxstyleindexentry{general}\sphinxstyleindexpageref{general:\detokenize{module-general}}
\indexspace
\bigletter{k}
\item\relax\sphinxstyleindexentry{KNN\_sel\_var}\sphinxstyleindexpageref{KNN_sel_var:\detokenize{module-KNN_sel_var}}
\indexspace
\bigletter{m}
\item\relax\sphinxstyleindexentry{methode\_fait\_maison}\sphinxstyleindexpageref{methode_fait_maison:\detokenize{module-methode_fait_maison}}
\end{sphinxtheindex}

\renewcommand{\indexname}{Index}
\printindex
\end{document}