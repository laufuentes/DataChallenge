%% Generated by Sphinx.
\def\sphinxdocclass{report}
\documentclass[letterpaper,10pt,french]{sphinxmanual}
\ifdefined\pdfpxdimen
   \let\sphinxpxdimen\pdfpxdimen\else\newdimen\sphinxpxdimen
\fi \sphinxpxdimen=.75bp\relax
\ifdefined\pdfimageresolution
    \pdfimageresolution= \numexpr \dimexpr1in\relax/\sphinxpxdimen\relax
\fi
%% let collapsible pdf bookmarks panel have high depth per default
\PassOptionsToPackage{bookmarksdepth=5}{hyperref}

\PassOptionsToPackage{warn}{textcomp}
\usepackage[utf8]{inputenc}
\ifdefined\DeclareUnicodeCharacter
% support both utf8 and utf8x syntaxes
  \ifdefined\DeclareUnicodeCharacterAsOptional
    \def\sphinxDUC#1{\DeclareUnicodeCharacter{"#1}}
  \else
    \let\sphinxDUC\DeclareUnicodeCharacter
  \fi
  \sphinxDUC{00A0}{\nobreakspace}
  \sphinxDUC{2500}{\sphinxunichar{2500}}
  \sphinxDUC{2502}{\sphinxunichar{2502}}
  \sphinxDUC{2514}{\sphinxunichar{2514}}
  \sphinxDUC{251C}{\sphinxunichar{251C}}
  \sphinxDUC{2572}{\textbackslash}
\fi
\usepackage{cmap}
\usepackage[T1]{fontenc}
\usepackage{amsmath,amssymb,amstext}
\usepackage{babel}



\usepackage{tgtermes}
\usepackage{tgheros}
\renewcommand{\ttdefault}{txtt}



\usepackage[Sonny]{fncychap}
\ChNameVar{\Large\normalfont\sffamily}
\ChTitleVar{\Large\normalfont\sffamily}
\usepackage{sphinx}

\fvset{fontsize=auto}
\usepackage{geometry}


% Include hyperref last.
\usepackage{hyperref}
% Fix anchor placement for figures with captions.
\usepackage{hypcap}% it must be loaded after hyperref.
% Set up styles of URL: it should be placed after hyperref.
\urlstyle{same}

\addto\captionsfrench{\renewcommand{\contentsname}{Contents:}}

\usepackage{sphinxmessages}
\setcounter{tocdepth}{1}



\title{Fonctions Regression}
\date{nov. 09, 2023}
\release{2023}
\author{Benali Nafissa et Fuentes Vicente Laura}
\newcommand{\sphinxlogo}{\vbox{}}
\renewcommand{\releasename}{Version}
\makeindex
\begin{document}

\ifdefined\shorthandoff
  \ifnum\catcode`\=\string=\active\shorthandoff{=}\fi
  \ifnum\catcode`\"=\active\shorthandoff{"}\fi
\fi

\pagestyle{empty}
\sphinxmaketitle
\pagestyle{plain}
\sphinxtableofcontents
\pagestyle{normal}
\phantomsection\label{\detokenize{index::doc}}


\sphinxstepscope


\chapter{combinaison module}
\label{\detokenize{combinaison:combinaison-module}}\label{\detokenize{combinaison::doc}}
\sphinxstepscope


\chapter{dataset\_division module}
\label{\detokenize{dataset_division:module-dataset_division}}\label{\detokenize{dataset_division:dataset-division-module}}\label{\detokenize{dataset_division::doc}}\index{module@\spxentry{module}!dataset\_division@\spxentry{dataset\_division}}\index{dataset\_division@\spxentry{dataset\_division}!module@\spxentry{module}}\index{formal\_div() (dans le module dataset\_division)@\spxentry{formal\_div()}\spxextra{dans le module dataset\_division}}

\begin{fulllineitems}
\phantomsection\label{\detokenize{dataset_division:dataset_division.formal_div}}
\pysigstartsignatures
\pysiglinewithargsret{\sphinxcode{\sphinxupquote{dataset\_division.}}\sphinxbfcode{\sphinxupquote{formal\_div}}}{\emph{\DUrole{n}{data}}, \emph{\DUrole{n}{idx0}}, \emph{\DUrole{n}{idx1}}}{}
\pysigstopsignatures
\sphinxAtStartPar
Fonction qui sépare le jeu de données data en deux jeux de données data0 (indices où wine\_type=0) et data1 (resp. wine\_type=1)
\begin{quote}\begin{description}
\sphinxlineitem{Paramètres}\begin{itemize}
\item {} 
\sphinxAtStartPar
\sphinxstyleliteralstrong{\sphinxupquote{data}} (\sphinxstyleliteralemphasis{\sphinxupquote{pd.DataFrame}}) \textendash{} Jeu de données à séparer en data0 et data1

\item {} 
\sphinxAtStartPar
\sphinxstyleliteralstrong{\sphinxupquote{idx0}} (\sphinxstyleliteralemphasis{\sphinxupquote{np.array}}) \textendash{} Indices où wine\_type=0

\item {} 
\sphinxAtStartPar
\sphinxstyleliteralstrong{\sphinxupquote{idx1}} (\sphinxstyleliteralemphasis{\sphinxupquote{\_type\_}}) \textendash{} Indices où wine\_type=1

\end{itemize}

\sphinxlineitem{Renvoie}
\sphinxAtStartPar
Jeu de données contenant les vins de type 0
data1 (pd.DataFrame): Jeu de données contenant les vins de type 1

\sphinxlineitem{Type renvoyé}
\sphinxAtStartPar
data0 (pd.DataFrame)

\end{description}\end{quote}

\end{fulllineitems}

\index{treatment() (dans le module dataset\_division)@\spxentry{treatment()}\spxextra{dans le module dataset\_division}}

\begin{fulllineitems}
\phantomsection\label{\detokenize{dataset_division:dataset_division.treatment}}
\pysigstartsignatures
\pysiglinewithargsret{\sphinxcode{\sphinxupquote{dataset\_division.}}\sphinxbfcode{\sphinxupquote{treatment}}}{\emph{\DUrole{n}{data}}}{}
\pysigstopsignatures
\sphinxAtStartPar
Fonction qui sépare un dataset donné en co\sphinxhyphen{}variables (X) et variable explicative (y).
NB: cette fonction serà implementée que sur les jeu de données train (data0, data1)
\begin{quote}\begin{description}
\sphinxlineitem{Paramètres}
\sphinxAtStartPar
\sphinxstyleliteralstrong{\sphinxupquote{data}} (\sphinxstyleliteralemphasis{\sphinxupquote{pd.DataFrame}}) \textendash{} Jeu de données à séparer X, y

\sphinxlineitem{Renvoie}
\sphinxAtStartPar
Co\sphinxhyphen{}variables du jeu de données
y (pd.DataFrame): Variable explicative

\sphinxlineitem{Type renvoyé}
\sphinxAtStartPar
X (pd.DataFrame)

\end{description}\end{quote}

\end{fulllineitems}

\index{winetype() (dans le module dataset\_division)@\spxentry{winetype()}\spxextra{dans le module dataset\_division}}

\begin{fulllineitems}
\phantomsection\label{\detokenize{dataset_division:dataset_division.winetype}}
\pysigstartsignatures
\pysiglinewithargsret{\sphinxcode{\sphinxupquote{dataset\_division.}}\sphinxbfcode{\sphinxupquote{winetype}}}{\emph{\DUrole{n}{data}}}{}
\pysigstopsignatures
\sphinxAtStartPar
Fonction qui repère les indices d’un dataset où la variable « wine\_type » est 0 et 1.
\begin{quote}\begin{description}
\sphinxlineitem{Paramètres}
\sphinxAtStartPar
\sphinxstyleliteralstrong{\sphinxupquote{data}} (\sphinxstyleliteralemphasis{\sphinxupquote{pd.DataFrame}}) \textendash{} Jeu de données à l’origine de la division en fonction de la variable « wine\_type »

\sphinxlineitem{Renvoie}
\sphinxAtStartPar
vecteur contenant les indices des lignes telles que « wine\_type »=0
index\_1 (np.array): vecteur contenant les indices des lignes telles que « wine\_type »=1

\sphinxlineitem{Type renvoyé}
\sphinxAtStartPar
index\_0 (np.array)

\end{description}\end{quote}

\end{fulllineitems}


\sphinxstepscope


\chapter{general module}
\label{\detokenize{general:module-general}}\label{\detokenize{general:general-module}}\label{\detokenize{general::doc}}\index{module@\spxentry{module}!general@\spxentry{general}}\index{general@\spxentry{general}!module@\spxentry{module}}\index{CV\_rep() (dans le module general)@\spxentry{CV\_rep()}\spxextra{dans le module general}}

\begin{fulllineitems}
\phantomsection\label{\detokenize{general:general.CV_rep}}
\pysigstartsignatures
\pysiglinewithargsret{\sphinxcode{\sphinxupquote{general.}}\sphinxbfcode{\sphinxupquote{CV\_rep}}}{\emph{\DUrole{n}{Xtr}}, \emph{\DUrole{n}{ytr}}, \emph{\DUrole{n}{nfolds}}}{}
\pysigstopsignatures
\sphinxAtStartPar
La fonction crée des nouveaux jeux de données en sous\sphinxhyphen{}divisant les jeux de données en
\begin{quote}\begin{description}
\sphinxlineitem{Paramètres}\begin{itemize}
\item {} 
\sphinxAtStartPar
\sphinxstyleliteralstrong{\sphinxupquote{Xtr}} (\sphinxstyleliteralemphasis{\sphinxupquote{pd.DataFrame}}) \textendash{} Jeu de données à diviser, contenant les co\sphinxhyphen{}variables

\item {} 
\sphinxAtStartPar
\sphinxstyleliteralstrong{\sphinxupquote{ytr}} (\sphinxstyleliteralemphasis{\sphinxupquote{pd.DataFrame}}) \textendash{} Vecteur à diviser, contenant la variable à prédire

\item {} 
\sphinxAtStartPar
\sphinxstyleliteralstrong{\sphinxupquote{nfolds}} (\sphinxstyleliteralemphasis{\sphinxupquote{int}}) \textendash{} Nombre representant en combien de sous dataframes on souhaite diviser Xtr et ytr

\end{itemize}

\sphinxlineitem{Renvoie}
\sphinxAtStartPar
liste contenant (n\sphinxhyphen{}folds) jeux de données
y\_new (list): liste contenant (les n\sphinxhyphen{}folds) nouvelles version de ytr

\sphinxlineitem{Type renvoyé}
\sphinxAtStartPar
X\_new (list)

\end{description}\end{quote}

\end{fulllineitems}

\index{build\_pred() (dans le module general)@\spxentry{build\_pred()}\spxextra{dans le module general}}

\begin{fulllineitems}
\phantomsection\label{\detokenize{general:general.build_pred}}
\pysigstartsignatures
\pysiglinewithargsret{\sphinxcode{\sphinxupquote{general.}}\sphinxbfcode{\sphinxupquote{build\_pred}}}{\emph{\DUrole{n}{X\_test0}}, \emph{\DUrole{n}{X\_test1}}, \emph{\DUrole{n}{pred0}}, \emph{\DUrole{n}{pred1}}}{}
\pysigstopsignatures
\sphinxAtStartPar
Fonction qui combine les prédictions de data0 et data1
\begin{quote}\begin{description}
\sphinxlineitem{Paramètres}\begin{itemize}
\item {} 
\sphinxAtStartPar
\sphinxstyleliteralstrong{\sphinxupquote{X\_test0}} (\sphinxstyleliteralemphasis{\sphinxupquote{pd.DataFrame}}) \textendash{} Vecteur contenant les co\sphinxhyphen{}variables à wine\_type=0 du jeu de données test

\item {} 
\sphinxAtStartPar
\sphinxstyleliteralstrong{\sphinxupquote{X\_test1}} (\sphinxstyleliteralemphasis{\sphinxupquote{pd.DataFrame}}) \textendash{} Vecteur contenant les co\sphinxhyphen{}variables à wine\_type=0 du jeu de données test

\item {} 
\sphinxAtStartPar
\sphinxstyleliteralstrong{\sphinxupquote{pred0}} (\sphinxstyleliteralemphasis{\sphinxupquote{np.array ou pd.DataFrame}}) \textendash{} prédictions associés pour les indices de wine\_type=0

\item {} 
\sphinxAtStartPar
\sphinxstyleliteralstrong{\sphinxupquote{pred1}} (\sphinxstyleliteralemphasis{\sphinxupquote{np.array ou pd.DataFrame}}) \textendash{} prédictions associés pour les indices de wine\_type=1

\end{itemize}

\sphinxlineitem{Renvoie}
\sphinxAtStartPar
Vecteur contenant l’ensemble des prédictions du jeu de données test

\sphinxlineitem{Type renvoyé}
\sphinxAtStartPar
pred (np.array)

\end{description}\end{quote}

\end{fulllineitems}

\index{param\_selection() (dans le module general)@\spxentry{param\_selection()}\spxextra{dans le module general}}

\begin{fulllineitems}
\phantomsection\label{\detokenize{general:general.param_selection}}
\pysigstartsignatures
\pysiglinewithargsret{\sphinxcode{\sphinxupquote{general.}}\sphinxbfcode{\sphinxupquote{param\_selection}}}{\emph{\DUrole{n}{param}}, \emph{\DUrole{n}{mod}}, \emph{\DUrole{n}{Xtr}}, \emph{\DUrole{n}{ytr}}, \emph{\DUrole{n}{Xte}}}{}
\pysigstopsignatures
\sphinxAtStartPar
\_summary\_
\begin{quote}\begin{description}
\sphinxlineitem{Paramètres}\begin{itemize}
\item {} 
\sphinxAtStartPar
\sphinxstyleliteralstrong{\sphinxupquote{param}} (\sphinxstyleliteralemphasis{\sphinxupquote{liste}}) \textendash{} \{« param1 »:{[}“option1”, “option2”{]}, »param2 »:{[}option1,option2,option3{]}\}

\item {} 
\sphinxAtStartPar
\sphinxstyleliteralstrong{\sphinxupquote{mod}} (\sphinxstyleliteralemphasis{\sphinxupquote{sklearn.function}}) \textendash{} model(random\_state=10)

\item {} 
\sphinxAtStartPar
\sphinxstyleliteralstrong{\sphinxupquote{Xtr}} (\sphinxstyleliteralemphasis{\sphinxupquote{pd.DataFrame}}) \textendash{} Co\sphinxhyphen{}variables du jeu de données train

\item {} 
\sphinxAtStartPar
\sphinxstyleliteralstrong{\sphinxupquote{ytr}} (\sphinxstyleliteralemphasis{\sphinxupquote{pd.DataFrame}}) \textendash{} Variable à prédire du jeu de données train

\item {} 
\sphinxAtStartPar
\sphinxstyleliteralstrong{\sphinxupquote{Xte}} (\sphinxstyleliteralemphasis{\sphinxupquote{pd.DataFrame}}) \textendash{} Co\sphinxhyphen{}variables du jeu de données test

\end{itemize}

\sphinxlineitem{Renvoie}
\sphinxAtStartPar
vecteur contenant les prédictions du modèle selectionné par cross validation

\sphinxlineitem{Type renvoyé}
\sphinxAtStartPar
pred (np.array ou pd.DataFrame)

\end{description}\end{quote}

\end{fulllineitems}

\index{soumission() (dans le module general)@\spxentry{soumission()}\spxextra{dans le module general}}

\begin{fulllineitems}
\phantomsection\label{\detokenize{general:general.soumission}}
\pysigstartsignatures
\pysiglinewithargsret{\sphinxcode{\sphinxupquote{general.}}\sphinxbfcode{\sphinxupquote{soumission}}}{\emph{\DUrole{n}{pred}}, \emph{\DUrole{n}{date}}, \emph{\DUrole{n}{name\_pred}}}{}
\pysigstopsignatures
\sphinxAtStartPar
Fonction qui crée une soumission avec le nom de la methode et la sauvagarde dans datasets\_c
\begin{quote}\begin{description}
\sphinxlineitem{Paramètres}\begin{itemize}
\item {} 
\sphinxAtStartPar
\sphinxstyleliteralstrong{\sphinxupquote{pred}} (\sphinxstyleliteralemphasis{\sphinxupquote{np.array}}) \textendash{} prediction

\item {} 
\sphinxAtStartPar
\sphinxstyleliteralstrong{\sphinxupquote{date}} (\sphinxstyleliteralemphasis{\sphinxupquote{str}}) \textendash{} date ex: “0110”

\item {} 
\sphinxAtStartPar
\sphinxstyleliteralstrong{\sphinxupquote{name\_pred}} (\sphinxstyleliteralemphasis{\sphinxupquote{str}}) \textendash{} nom de la méthode utilisée

\end{itemize}

\end{description}\end{quote}

\end{fulllineitems}

\index{train\_eval() (dans le module general)@\spxentry{train\_eval()}\spxextra{dans le module general}}

\begin{fulllineitems}
\phantomsection\label{\detokenize{general:general.train_eval}}
\pysigstartsignatures
\pysiglinewithargsret{\sphinxcode{\sphinxupquote{general.}}\sphinxbfcode{\sphinxupquote{train\_eval}}}{\emph{\DUrole{n}{model}}, \emph{\DUrole{n}{X}}, \emph{\DUrole{n}{y}}, \emph{\DUrole{n}{X\_test}}, \emph{\DUrole{n}{y\_test}}}{}
\pysigstopsignatures
\sphinxAtStartPar
Fonction qui entraine un modèle, plotte et renvoit le r2\_score sur le jeu de données test associé à la prédiction
\begin{quote}\begin{description}
\sphinxlineitem{Paramètres}\begin{itemize}
\item {} 
\sphinxAtStartPar
\sphinxstyleliteralstrong{\sphinxupquote{model}} (\sphinxstyleliteralemphasis{\sphinxupquote{sklearn.function}}) \textendash{} Modèle à entrainer pour faire la prediction

\item {} 
\sphinxAtStartPar
\sphinxstyleliteralstrong{\sphinxupquote{X}} (\sphinxstyleliteralemphasis{\sphinxupquote{pd.Dataframe}}) \textendash{} Co\sphinxhyphen{}variables du jeu de données train

\item {} 
\sphinxAtStartPar
\sphinxstyleliteralstrong{\sphinxupquote{y}} (\sphinxstyleliteralemphasis{\sphinxupquote{\_type\_}}) \textendash{} Variable à prédire du jeu de données train

\item {} 
\sphinxAtStartPar
\sphinxstyleliteralstrong{\sphinxupquote{X\_test}} (\sphinxstyleliteralemphasis{\sphinxupquote{\_type\_}}) \textendash{} Co\sphinxhyphen{}variables du jeu de données test

\item {} 
\sphinxAtStartPar
\sphinxstyleliteralstrong{\sphinxupquote{y\_test}} (\sphinxstyleliteralemphasis{\sphinxupquote{\_type\_}}) \textendash{} Variable à prédire du jeu de données test

\end{itemize}

\sphinxlineitem{Renvoie}
\sphinxAtStartPar
prédiction sur le jeu de données test associé au modèle choisi en input

\sphinxlineitem{Type renvoyé}
\sphinxAtStartPar
pred (np.array ou pd.DataFrame)

\end{description}\end{quote}

\end{fulllineitems}



\chapter{Indices and tables}
\label{\detokenize{index:indices-and-tables}}\begin{itemize}
\item {} 
\sphinxAtStartPar
\DUrole{xref,std,std-ref}{genindex}

\item {} 
\sphinxAtStartPar
\DUrole{xref,std,std-ref}{modindex}

\item {} 
\sphinxAtStartPar
\DUrole{xref,std,std-ref}{search}

\end{itemize}


\renewcommand{\indexname}{Index des modules Python}
\begin{sphinxtheindex}
\let\bigletter\sphinxstyleindexlettergroup
\bigletter{d}
\item\relax\sphinxstyleindexentry{dataset\_division}\sphinxstyleindexpageref{dataset_division:\detokenize{module-dataset_division}}
\indexspace
\bigletter{g}
\item\relax\sphinxstyleindexentry{general}\sphinxstyleindexpageref{general:\detokenize{module-general}}
\end{sphinxtheindex}

\renewcommand{\indexname}{Index}
\printindex
\end{document}